\documentclass{s0minutes}
\usepackage[utf8]{inputenc}
\usepackage[ngerman]{babel}
\usepackage{longtable}
\usepackage{booktabs} % professional tables
\usepackage{multicol}
\usepackage{wasysym}  % for \diameter
\usepackage{textcomp} % for €

\meetingminutes{\generalassembly}{14. Januar 2018}{14:00}{Stratum 0,
Braunschweig}{32 stimmberechtigte Mitglieder,\\ & keine nicht stimmberechtigten
Mitglieder,\\ & ein Gast}{}{rohieb}

\title{9.\, Mitgliederversammlung}

\begin{document}
\maketitle

%%%%%%%%%%%%
%% TOP 0  %%
%%%%%%%%%%%%
\section{Protokoll-Overhead}
\begin{description}
  \item[Eröffnung der Versammlung] durch den Vorstandsvorsitzenden um 14:15
  \item[Wahl der Versammlungsleitung:] Kasa, kein Einspruch
  \item[Wahl der Protokollführung:] rohieb, kein Einspruch
  \item[Quoren:] zum Tag der Mitgliederversammlung hat der Verein insgesamt 102
    Mitglieder, davon 92 ordentliche Mitglieder.
    \begin{itemize}
      \item 21{,}16 Mitglieder = 23\% der ordentlichen Mitglieder für
        Beschlussfähigkeit
      \item 16 Mitglieder = 50\% der anwesenden, stimmberechtigten Mitglieder
        für Annahme eines Antrags
    \end{itemize}
  \item[Beschlussfähigkeit:] 32 von geforderten 21{,}16 (23\%) stimmberechtigten
    Mitglieder anwesend, die Versammlung ist damit beschlussfähig.
  \item[Notation für Abstimmungen:] (Pro-Stimmen/Contra-Stimmen/Enthaltungen)
\end{description}

%%%%%%%%%%%%%
%% TOP 1   %%
%%%%%%%%%%%%%
\section{Berichte}

%%%%%%%%%%%%%
%% TOP 1.1 %%
%%%%%%%%%%%%%
\subsection{Jahres(abschnitts)bericht}

larsan und chrissi\^{} geben einen Überblick über das vergangene Jahr.

Die Gemeinnützigkeit wurde dem Verein vom Finanzamt bestätigt, die nächste Prüfung findet 2019 statt.

\paragraph{Gruppen und Veranstaltungen}

Im Space finden regelmäßig Vorträge statt, im vergangenen Jahr 35 Stück, und seit Gründung insgesamt 215 Vorträge. In den letzten Monaten wurden die Vorträge in der Regel auch aufgezeichnet. Weitere Verbesserungen des Aufzeichnungs-Workflows sind geplant. larsan weist darauf hin, dass später am Abend noch Vorträge stattfinden.

Insgesamt fanden im Space 25 CoderDojos statt. In Zukunft werden diese immer am vorletzten Samstag im Monat stattfinden, Mentoren sind immer gerne gesehen. Allgemeines Vorwissen ist dabei kaum vonnöten. Im november gab es einen Artikel in der Braunschweiger Zeitung, seitdem sind die Veranstaltungen ausgebucht und raummäßig am Limit. Bei interesse kann auch nochmal eine Mentorenvorbereitungsrunde stattfinden. Das nächste CoderDojo ist am nächsten Samstag.

Das lokale Freifunk-Projekt wird momentan rebootet.

Im Rahmen des Labdoo-Projektes werden Laptops gesammelt und in infrastrukturschwache Länder verschifft.

Die Braunschweiger Linux-User-Group hat sich im Sommer bei uns getroffen, sie sind jetzt ins Haus der Talente in der Weststadt umgezogen.

Digitalcourage trifft sich jeden zweiten Donnerstag im Space, und plant demnächst eine Cryptoparty.

Das Malkränzchen, ein offener Kreativabend, findet jeden Montagabend statt.

Der Braunschweiger Kopter Club trifft sich jeden zweiten Dienstag zum Stammtisch, und besteht aus relativ vielen Leuten. Sie haben auch wieder Geld gespendet dieses Jahr.

Das Animereferat pausiert momentan, Captain's Log findet regelmäßig statt.

Das Stratum 0 Phone Operation Center hat jetzt eine eigene DECT-Basestation passend zur gespendeten Alcatel-Telefonanlage. Diese läuft wegen des hohen Stromverbrauchs nicht permanent, kann aber bei Veranstaltungen eingesetzt werden.

Die Vegan Academy wurde inzwischen auch vom Backspace in Bamberg übernommen.

\paragraph{Öffentlichkeitsarbeit}

Stratum 0 war auf dem Markt der Kreativen vertreten. Auf der Maker Faire Hannover dieses Jahr hingegen nicht.

\paragraph{Infrastruktur}

Ein neuer Router, ein PC-Engines APU2C4 wurde angeschafft, dieser betreibt ein Test-WiFi mit UniFi-Accesspoints.
Wir sind nach wie vor Teil des Gründerquartieres der Stadt Braunschweig.
Freifunk wurde mit 4500€ durch die Stadt Braunschweig gefördert, das Geld soll für ein eigenes Event-WLAN-Setup aus 16 Accesspoints ausgegeben werden, und es ist eine continouos integration für Freifunkrouter geplant.

\paragraph{Community}

Mitglieder waren auf dem Easterhegg, der GPN, dem SHA, und dem Hackover. Das Hacken Open Air wurde von uns selbst organisiert, larsan zeigt Fotos: Zelt, Zelt, Zelt, Zelt, Teppich, Dusche, tolle Dusche. Es warem am Ende insgesamt etwa 130 Leute anwesend, das Wetter war durchwachsen. Es ist geplant, ein solches Event zu wiederholen, Details werden demnächst in einem Orgatreffen ausgearbeitet.

Auf dem 34C3 war Stratum 0 wieder mit Mensadisplay, wir haben 150 Ticket-Vouchers generiert. larsan schläft vor, nächstes Jahr früher mit der Planung zu beginnen - dafür ist es nicht erforderlich, Teil des Vorstandes zu sein.

Für den nächsten Congress gibt es die Überlegung, mit anderen unabhängigen Hackerspaces einen eigenen "Orbit" zu gründen.

An Silvester waren 20 Entitäten im Space, es wurde über Brandschutz diskutiert und Sicherungen getestet.

\paragraph{Diverses}

Das neues LED-Licht fällt nicht mehr in der Stromstatistik auf.
An Spacebauabenden wurden diverse Infrastrukturdinge gehackt.
Das Shenzhen-Portal wächst.
Stratumnews werden weitergeführt.
Es fanden Parties für Enno und Marudor statt.
Es wurden Bier, Wurst, Sushi, und weiteres Essen und Trinken hergestellt.
Die Spendendose ist verschwunden, es wurde Anzeige erstattet und eine neue gebaut (mit Kensington-Schloss).
Selgros- und Metrokarten sind für Mitglieder verfügbar.

\paragraph{Zukunft und Ausblick}

Nach verbreiteter Wahrnehmung ist der Space zu klein, und das Problem wächst.

Die Space-Suche in der Umgebung gestaltet sich schwierig. Der Serverraum von gaertner und das Archiv neben uns sind nicht verfügbar, die Räumlichkeiten im C3-Gebäudeteil des Schimmelhofs wurden plötzlich vermietet, die IEK unten im Haus ist nach Wasserschaden umgezogen, deren Räume werden aber vermutlich vom Staatstheather genutzt werden, und kosten 9€ pro Quadratmeter, was momentan nicht diskutabel scheint. Der Wasserschaden wird von der Versicherung übernommen, und gilt nicht als Mangel. Laut Kontakt mit der Verwaltung gibt es im Schimmelhof gerade keine geeigneten Räume.

Die Werkstatt bekam eine Festool-Sachspende, außerdem wurden ein Schweißgerät von EWM, eine Standbohrmaschine, eine Akku-Stichsäge und ein Akkubohrer angeschafft.

Die Selbstsicht des Vorstandes wird erläutert: Er sieht sich eher Verwaltungsorgan, nicht als Regierung, sondern eher als erweiterter Briefkasten, alles andere sollte aus der Basis kommen.
Die Mitglieder werden aufgefordert, im Space coole Dinge zu tun, und nicht immer den Vorstand entscheiden zu lassen. Es gibt regelmäßige Arbeitstreffen zur Geschäftsführung, das prinzipiell offen für alle Mitglieder ist.

tooling

kontakte mit firmen/sponsoren

(screenshots von twitter mit testimonials)

\pagagraph{Sponsoren}

siehe wiki, etwas mehr als letztes jahr :)

%%%%%%%%%%%%%
%% TOP 1.2 %%
%%%%%%%%%%%%%
\subsection{Finanzbericht}

%%%%%%%%%%%%%
%% TOP 1.3 %%
%%%%%%%%%%%%%
\subsection{Bericht der Rechnungsprüfer}

%%%%%%%%%%%%%
%% TOP 2   %%
%%%%%%%%%%%%%
\section{Entlastung des Vorstands}

Bei der letzten Mitgliederversammlung wurde der Vorstand nicht entlastet, die Probleme wurden aber behoben. Auf die Frage hin, ob jemand möchte, dass die Vorstände einzeln entlastet werden, meldet sich niemand.

Die Versammlungsleitung bittet um Entlastung des Vortandes für die gesamte Vergangenheit, die noch nicht entlastet wurde (die letzten 13 Monate). Es wird per Handzeichen abgestimmt.

\begin{resolution}{MV 2017-01}{\vote{\adopted}{28}{0}{6}}{Entlastung des
  Vorstandes}{}
  Alle Vorstandsmitglieder enthalten sich.
\end{resolution}

Der Vorstand ist damit für die letzten 13 Monate entlastet.

%%%%%%%%%%%%%
%% TOP 3   %%
%%%%%%%%%%%%%
\section{Wahlen}

Als Wahlleitung wird gecko durch Handzeichen ohne Gegenstimmen oder
Enthaltungen gewählt. Die Versammlungsleitung wird an die Wahlleitung übergeben.
Die Wahlleitung erklärt den Wahlmodus: Approval voting mit Stimmzetteln. Für jede Entität auf dem Wahlzettel darf ein Kreuz oder kein Kreuz gesetzt werden.
Es gibt ein Quorum von 50\%: Die Person, die das Quorum erreicht und die meisten Stimmen hat, ist gewählt.
Bei Beisitzern mit mehreren Posten, werden alle Personen, die das Quorum schaffen, in absteigender Reihenfolge der Stimmen besetzt.
Falls niemand das Quorum schafft, findet eine Nachwahl für die notwendigen Posten (die Vorsitzenden und den Schatzmeister) statt.

Für alle Posten findet ein gemeinsamer Wahlgang statt, dann wird von "oben" nach "unten" besetzt: Wer als 1. Vorsitzender schon gewählt wurde, fällt für den 2. Vorsitzenden weg, ein gewählter 2. Vorsitzender fällt für den Posten des Schatzmeisters weg, ein gewählter Schatzmeister fällt für die Posten des Beisitzers weg.

Um sich zu enthalten, wird der Stimmzettel nicht abgegeben. Ein großes kreuz über den gesamten Block eines einzelnen Vorstandspostens wird als Enthaltung für diesen Posten gewertet.

%%%%%%%%%%%%%
%% TOP 4   %%
%%%%%%%%%%%%%
\section{Sonstiges}

\meetingend{16:37}
\end{document}

% vim: set et ts=2 sw=2 sts=2 :
