\documentclass[12pt,a4paper]{scrartcl}

\usepackage[ngerman]{babel}
\usepackage[T1]{fontenc}
\usepackage[utf8]{inputenc}
\usepackage[legal]{stratum0doc}
\usepackage{libertine}    % nicht unbedingt notwendig

\title{The Hackerspace Agreement}
\date{5.~Februar~2012}

\begin{document}
\maketitle

\noindent Mit der Nutzung des Hackerspaces erkennen die benutzenden Entitäten
folgende Ordnung an.

\section{Allgemeines}
\begin{enumerate}
  \item Alle Einrichtungsgegenstände, elektronischen Geräte und vor Ort
    befindlichen Entitäten im Space sind pfleglich zu behandeln.

  \item Im gesamten Space gilt ein Rauchverbot für alle rauchbaren und nicht
    rauchbaren Substanzen. Ausgenommen davon sind elektronische Bauteile, die
    aus eigenem Antrieb zu rauchen anfangen.

  \item Aufräumende oder putzende Entitäten werden als Putz- oder Aufräumentität
    bezeichnet. Putz- oder Aufräumentitäten genießen Heldenstatus und haben im
    Rahmen ihrer Putz- oder Aufräumtätigkeit Weisungsrecht gegenüber
    nichtaufräumenden und nichtputzenden Entitäten.

  \item\begin{enumerate}
    \item Offensichtlich besitzerlose Getränke sind Allgemeingut und dürfen ohne
      Rückfrage entsorgt werden. Um dies zu verhindern, kann das Getränk vor Ort
      vom Besitzer gekennzeichnet werden, sodass der Besitzer von anderen
      Entitäten leicht erkennbar ist. Hierzu sei angemerkt, dass insbesondere
      die Bezeichnungen "`Club-Mate"', "`Cola"', "`Fanta"' und andere
      Getränkemarken (in allen möglichen Abwandlungen) nicht als Besitz
      implizierende Kennzeichnung angesehen werden.
    \item Getränke, die offensichtlich längere Zeit unangetastet stehen, sind
      als besitzerlos anzusehen.
    \item Entitäten, die besitzerlose oder leere Getränke entsorgen, sind als
      Putz- oder Aufräumentität anzusehen.
  \end{enumerate}

  \item\begin{enumerate}
    \item Falls Kleinigkeiten (Toilettenpapier, Büromaterial, Müllbeutel, etc.)
      fehlen, werden sie auf die entsprechende Liste im Flur geschrieben.
    \item Falls sich eine nichtleere Menge Kleinigkeiten auf der entsprechenden
      Liste im Flur befindet, dürfen diese von jeder Entität angeschafft werden,
      die sich dafür bereit erklärt.
    \item Entitäten, die sich bereit erklären, Kleinigkeiten für Vereinszwecke
      anzuschaffen, werden als Einkaufsentitäten bezeichnet. Einkaufsentitäten
      genießen Heldenstatus.
    \item Die Erstattung von Beträgen, die von Einkaufsentitäten für
      Vereinszwecke ausgelegt wurden, erfolgt beim Schatzmeister gegen Vorlage
      eines Kassenbons, auf dem sich nur die für Vereinszwecke angeschafften
      Kleinigkeiten befinden dürfen.
    \item Die selbstständige Erstattung von Beträgen aus der Mate-Kasse ist
      ausdrücklich verboten!
  \end{enumerate}

  \item\begin{enumerate}
    \item Mitgliedern steht es frei, im Rahmen ihrer Tätigkeit an Projekten im
      Space Hardwareobjekte jeglicher Form dort aufzubewahren, sofern Platz zur
      Aufbewahrung vorhanden ist.
    \item Im Space aufbewahrte Hardwareobjekte sollten mit Edding,
      Post-It, Dymo, o.\,ä. gekennzeichnet werden, sodass der Besitzer für alle
      Mitglieder erkennbar ist.
    \item Der Space soll jedoch keine Schrotthalde werden. Falls im Space
      aufbewahrte Hardwareobjekte allgemein als unerwünscht angesehen werden,
      kann der Besitzer vom Vorstand aufgefordert werden, diese Objekte aus dem
      Space zu entfernen.
    \item Geschieht die Entfernung nicht innerhalb eines Monats nach der
      Aufforderung durch den Vorstand, kann der Vorstand das Hardwareobjekt in
      das Eigentum einer anderen Entität übergeben.
    \item Entitäten, die unerwünschte Hardwareobjekte entfernen, werden als
      Aufräumentitäten angesehen.
  \end{enumerate}

  \item Jede Entität, die den Space verlässt, wäscht vorher mindestens das von
    ihr benutzte Geschirr ab und bringt falls nötig auf dem Weg nach unten den
    Müll in die dafür vorgesehenen Müllcontainer auf dem Parkplatz. Falls
    Geschirr abgewaschen oder Müll entsorgt wird, ist die diese Aktion ausübende
    Entität als Putz- oder Aufräumentität anzusehen.

  \item Die letzte Entität, die den Space verlässt, dreht die Heizung herunter,
    schließt alle Fenster, schaltet alle Lampen und elektronischen Geräte aus,
    die nicht dauerhaft laufen müssen, und zieht die Tür hinter sich zu.
\end{enumerate}

\section{Küche}
\begin{enumerate}
  \item Die selbstständige Aneignung von Getränken durch anwesende Entitäten ist
    nach Zahlung des entsprechenden Preises in die Mate-Kasse gestattet.
  \item Nach Benutzung der Kaffeemaschine ist der gebrauchte Filter zu
    entsorgen.
  \item\begin{enumerate}
    \item Alle Kühlschrankinhalte sind grundsätzlich Allgemeingut und dürfen
      ohne Rückfrage vernichtet werden.
    \item An Kühlschrankinhalten kann eine Kennzeichnung (Post-It, Edding
      o.\,ä.) mit Name des Besitzers und Anfangsdatum der Lagerung
      angebracht werden. Dementsprechend gekennzeichnete Inhalte sind als
      Eigentum des Besitzers zu betrachten. Falls das Datum mehr als drei Tage
      zurückliegt, entfällt dieser Sonderstatus wieder.
  \end{enumerate}
\end{enumerate}

\section{Bad}
\begin{enumerate}
  \item Das Bad ist in einem sauberen Zustand zu halten. Dazu gehört, selbst
    verursachte Unsauberkeiten zu entfernen. Entitäten, die Unsauberkeiten im
    Bad entfernen, werden als Putzentitäten angesehen.

  \item Sitzpinkeln is mandatory. Bei Nichtbeachtung dieser Regelung wird
    Pinkelverbot ausgesprochen.
\end{enumerate}

\section{Übergangs- und Schlussbestimmungen}
\begin{enumerate}
  \item Schlafende Personen haben keinen Anspruch auf Ruhe oder Sonderbehandlung
    entsprechend dieser Verordnung.

  \item Die oben genannten Bestimmungen dieser Ordnung sind als unveränderlich
    anzusehen.

  \item Diese Ordnung tritt mit ihrer Verkündung durch den Vorstand in Kraft.
\end{enumerate}

\end{document}
% vim: set tw=80 et sw=2 ts=2:
