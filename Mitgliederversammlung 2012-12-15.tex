% vim:tw=80 ts=2 et sw=2 indentexpr= :
\documentclass[a4paper,12pt]{scrartcl}
\usepackage[utf8]{inputenc}
\usepackage[T1]{fontenc}
\usepackage[ngerman]{babel}
\usepackage{libertine} % kann man notfalls auch ignorieren, wenns nicht da ist
\usepackage{textcomp} % notfalls für €
\usepackage{stratum0doc}
\usepackage[colorlinks=false]{hyperref}
\usepackage{graphicx}
\usepackage{savefnmark}

\title{2.~Mitgliederversammlung des Stratum~0~e.~V.}
\date{15.~Dezember~2012}

\begin{document}
\addtokomafont{title}{\LARGE}
\maketitle
{\footnotesize\tableofcontents}

%%%%%%%%%%%
%% TOP 0 %%
%%%%%%%%%%%
\section{Eröffnung}
\begin{description}
  \item[Zeit:] 15. Dezember 2012, 14:00
  \item[Ort:] TanzSportZentrum Braunschweig, Hamburger Straße 273a
  \item[Anwesend:] 24 Mitglieder, davon 23 mit Stimmrecht (38{,}3\% von 60
    Mitgliedern insgesamt, Satzung fordert 23\%)
  \item[Wahl des Versammlungsleiters:] Vincent Breitmoser einstimmig durch
    Handzeichen; nimmt die Wahl an
  \item[Protokoll:] Roland Hieber einstimmig durch Handzeichen, nimmt die Wahl
    an.
  \item[Veranstaltung eröffnet] durch den Versammlungsleiter um 14:10
\end{description}

%%%%%%%%%%%
%% TOP 1 %%
%%%%%%%%%%%
\section{Berichte}
\subsection{Bericht des Vorstands}
Der Vorstandsvorsitzende berichtet stellvertretend für den Vorstand von den
Aktivitäten des Vorstands im vergangenen Jahr. Grundsätzlich war es wichtig, die
komplette Basis in seine Entscheidungen miteinzubeziehen. Dies geschah
insbesondere durch die Einführung von monatlichen Plena\footnote{für eine
Übersicht siehe \url{https://stratum0.org/wiki/Plenum}}, die zur Diskussion von
vereinsrelevanten Themen (u.~a. auch Zuweisung von Projektgeldern) dienten.

\emph{[Johannes Starosta erscheint verspätet und wird akkreditiert, insgesamt 24
anwesende stimmberechtigte Mitglieder.]}

\paragraph{Versicherung}
Ein großes Thema für den Vorstand dieses Jahr war das Abschließen einer
geeigneten Versicherung. Dazu wurden mehrere Angebote eingeholt (Allianz,
Gothaer, Öffentliche), außerdem wurden auch bei anderen Hackerspaces
Erkundigungen eingeholt. Das Thema ist bisher auch noch nicht abgeschlossen
worden, aber es wird wohl auf einen Betrag von 300-400€ im Jahr für Haftpflicht-
und Inhaltsversicherung hinauslaufen.

Es wird die Frage gestellt, was die Haftpflichtversicherung im Gewerbegebäude
absichert. Dies ist je nach Angebot verschieden, die Schadenssumme bewegt sich
aber bei den Angeboten für Vereinshaftpflicht standardmäßig im Rahmen von 2{,}5
bis 3 Millionen Euro.

Eine weitere Frage gestellt, welche Konditionen für Schlösser bei der
Inhaltsversicherung gilt. Zumindest beim Angebot der Allianz ist hier formal
gefordert, dass das Schloss von außen nicht abschraubbar ist (dies ist bei uns
gegeben), und dass eine Einbruchmeldeanlage installiert ist (ist bei uns nicht
der Fall, kommt aber auch mit größeren Kosten daher, der Vertreter von der
Allianz meinte auch, dass man in dem Punkt unter Umständen verhandeln könnte, da
solche Versicherungen primär auf freistehende Vereinsheime abzielen).

Ganz allgemein ist die Anzahl der Mitglieder eine Variable, die in die
Berechnung der Versicherungstarife eingeht. Dies ist jedoch nicht linear
gestaffelt.

Es wird außerdem angemerkt, dass bei den Angeboten zur Haftpflicht auch
Mitglieder versichert sind, sofern sie im Auftrag des Vereins stehen. In dieser
Hinsicht gibt es allerdings auch noch Angebote mit mehr Abdeckung, die aber für
uns nicht lukrativ sind.

Der Vorstandsvorsitzende erwähnt auch, dass der primäre Grund dafür, dass noch
keine Versicherung abgeschlossen wurde, ist, dass der Ansprechpartner bei der
Allianz im Moment aus privaten Gründen nicht verfügbar ist, sodass wir noch
keine Rückfragen zum Angebot stellen konnten. Die Mitglieder sollen aber darüber
informiert werden, wenn es in der Hinsicht Neuigkeiten gibt.

In Bezug auf eine Rechtsschutzversicherung wird der Nutzen diskutiert. Als
Beispiel dient der Fall, dass eine Person in den vereinseigenen Räumlichkeiten
über das Internet kinderpornographische oder ähnliche illegale Inhalte zur
Verfügung stellt. In diesem Fall wird grundsätzlich ein Strafverfahren eröffnet,
sodass ein Rechtsanwalt eingeschaltet werden muss. Im besten Fall läuft die
Strafe auf ein Bußgeld wegen Störerhaftung hinaus. Eine Rechtsschutzversicherung
würde zmindest die Kosten für solch einen Fall abfedern. Allerdings schlägt eine
solche Versicherung auch mit etwa 1000€ im Jahr zu Buche (ja nachdem, welche
Fälle abgedeckt sind). Dies entspricht auch etwa einer Abmahnung im Jahr. Es
wird angemerkt, dass man in solch einem Fall den Verein auch einfach auflösen
könnte, dies hätte jedoch bei einem strafrechtlichen Verfahren keinen Einfluss.
reneger bringt auch zur Sprache, dass es auch für Hackerspaces eine
Rechtsberatung gibt.

\paragraph{Fortbildungen}
Der Vorstandsvorsitzende führt weiter aus, dass er zusammen mit rohieb ein
(kostenloses) Seminar der Bürgerstiftung Braunschweig besucht hat, in dem die
Grundlagen über Vereinsrecht zusammengefasst wurden. Insbesondere haben sich
dadurch einige Fragen in Bezug auf die Gemeinnützigkeit geklärt, es wurden aber
auch andere Fragen aufgeworfen.

Ebenso haben rohieb und larsan ein Seminar über Öffentlichkeitsarbeit der
Bürgerstiftung besucht und ein paar wertvolle Punkte mitgenommen, die dem Verein
zu Gute kommen und in eine Arbeitsgruppe zur Öffentlichkeitsarbeit
einfließen könnten (siehe auch \ref{top:socialmedia} und
\ref{top:mitgliederwerbung})

\paragraph{Gemeinnützigkeit}
Die Gemeinnützigkeit des Vereins wird weiterhin aktiv angestrebt. Es gab
allerdings ziemlich lange Unklarheiten über die Vor- und Nachteile der
Gemeinnützigkeit, insbesondere, ob bzw. in welchem Rahmen Rechenschaft gegenüber
dem Finanzamt abgelegt werden muss. Als großer Vorteil wird weiterhin die
Steuerbegünstigung und die Möglichkeit zur Ausstellung von
Zuwendungsbescheinigungen (Spendenquittungen) gesehen, in bestimmten Fällen
können Mitglieder sogar ihre Mitgliedsbeiträge von der Steuer absetzen.
Nachteilig könnte sich unter Umständen auswirken, dass Spenden nur für
gemeinnützige Zwecke (laut Satzung) verwendet werden dürfen, und dass sie in der
Buchhaltung getrennt geführt werden müssen. Zudem sind auch viele andere
Hackerspaces in Deutschland gemeinnützig und steuerbegünstigt, wobei auch viele
ein Konstrukt aus Träger- bzw. Fördervereinen aufgebaut haben. Andere
Hackerspaces wiederum stehen der Gemeinnützigkeit kritisch gegenüber (so z.~B.
der CCC auf Bundesebene), weil sie meinen, dass man aus der Gemeinnützigkeit
nicht mehr so leicht herauskommt und sich angreifbar machen würde. Auch scheint
die Laune der lokalen Finanzämter eine große Rolle zu spielen.

Die Meinung des Vorstandsvorsitzenden (der sich hauptsächlich mit dem Thema
befasst hat) schwankt je nach Wissensstand sehr, aber es sollte seiner Meinung
nach schnell ein Beschluss in dieser Sache gefasst werden. Bisher ist der Verein
ein Jahr gut gelaufen, er würde aber noch die nächste Steuererklärung abwarten
wollen und dann auf deren Basis entscheiden.

\paragraph{Außenwirkung}
Da es in der letzten Zeit eine heftigere Diskussion um die Außenwirkung des
Vereins gab, wird dieser Punkt hier auch nochmal aufgegriffen. Grundsätzlich
konnten allerdings die meisten Kritikpunkte der Diskussion geklärt werden und es
gab mehrfach positives Feedback von außen. Dies könnte insbesondere durch unsere
Präsenz auf dem BarCamp Braunschweig im November 2012\footnote{siehe
\url{https://stratum0.org/wiki/BarCamp_Braunschweig_2012}} hervorgerufen worden
sein.

Weiterhin berichtet rohieb, dass mehrere Entitäten den Verein auf relevanten
Veranstaltungen bekannt gemacht haben, so z.~B. zur Eröffnung des Backspace in
Bamberg (März), zur SIGINT in Köln (Mai) und bei der TrollCon in Mannheim
(Oktober).

\paragraph{Sonstiges}
Die weiteren Aktivitäten des Vorstands beschränkten sich auf das übliche
Tagesgeschäft. Es wurden Gelder an Projekte verteilt (aus Basis der Empfehlungen
des monatlich stattfindenden Plenums der Mitglieder) und ein neuer Beamer
bestellt, der hauptsächlich aus dem Preisgeld unseres Teams beim rwthCTF
finanziert wurde. Außerdem wurde mit dem TanzSportZentrum verabredet, dass sie
unseren Internetanschluss benutzen dürfen (für etwa fünf YouTube-Videos im
Monat), im Gegenzug dürfen wir ihren Saal für die Mitgliederversammlung nutzen.

%%%%%%%%%%%
\subsection{Bericht des Schatzmeisters}
Der Schatzmeister hat eine Präsentation vorbereitet\footnote{siehe Anhang
\ref{sec:kassenbericht}} und berichtet, dass immer genug Geld vorhanden war. Die
größten Ausgaben waren Miete, Beamer und die neuen Küchenmöbel. Zum Zeitpunkt
der Kassenprüfung war auf dem Girokonto etwa 3141{,}09€ und in der Barkasse 712€
vorhanden, was insgesamt einen Betrag von 3853{,}09€ ausmacht. Des weiteren
drängt er dazu, mehr Geld auszugeben, bevor das Finanzamt was davon abhaben
will (dies könnte sich relativieren, sobald neue, größere Räumlichkeiten
gefunden sind, die Suche läuft nebenher, aber die Immobiliensituation ist nicht
besonders gut zur Zeit).

Zu seinem Aufruf, mehr Geld auszugeben, wird aus der Versammlung angemerkt,
dass mehr Projekte unter Umständen die Rücklagen aufbrauchen. Dies ist
grundsätzlich richtig, aber im Moment zahlen auch viele Mitglieder den
ermäßigten Beitrag, sodass noch etwas Potenzial nach oben vorhanden ist.
Außerdem sind die relevanteren Zahlen neben den Mitgliedsbeiträgen die laufenden
Kosten wie Miete und Nebenkosten. Im Moment sind ein Betrag um 2000€ als
Rücklagen eingeplant, die zur Deckung der laufenden Kosten dienen sollen für den
Fall, dass alle Mitglieder plötzlich austreten sollten. Voraussichtlich gibt es
bei dieser Summe auch kein Problem mit dem Finanzamt, in Bezug auf die
Gemeinnützigkeit ist Rücklagenbildung für einen bestimmten Zweck erlaubt. Da
dieser Betrag im Moment aber eher geschätzt ist, wird von einem Mitglied
vorgeschlagen, die absolut notwendigen Mindestkosten für den schlimmsten Fall
auszurechnen und bereitzuhalten.

Da der Schatzmeister im Moment auch die Mitgliederkartei führt, sagt er auch
noch ein paar Worte dazu. Es waren bei der Gründung im letzten Jahr 24
Entitäten Mitglied im Verein, seitdem sind 40 Mitglieder eingetreten und 3
Mitglieder ausgetreten. Außerdem wurde ein Mitglied vom Verein ausgeschlossen,
wobei das Vorgehen in dieser Hinsicht sehr liberal war (das Mitglied war seit
etwa einem Jahr nicht erreichbar und hatte auch keine Beiträge gezahlt). Auf die
Frage, warum Mitglieder ausgetreten sind, antwortet der Schatzmeister, dass dies
in fast allen Fällen wegen Umzug geschah.

%%%%%%%%%%%
\subsection{Bericht der Kassenprüfer}
Jan Lübbe berichtet als Kassenprüfer. Der Schatzmeister hat kein Geld
willentlich unterschlagen, allerdings war das nicht einfach herauszufinden. Wie
schon bei der letzten Mitgliederversammlung bemängelt er die Buchführung des
Schatzmeisters in mehreren Punkten. In Hinsicht auf Übersichtlichkeit sieht er
großes Verbesserungspotenzial nach oben, gerade auch, da es in Zukunft
gezwungernermaßen irgendwann Steuererklärungen geben wird.

Die Kassenprüfung ergab, dass 2{,}50€ in der Barkasse fehlten, die durch einen
Rabatt beim Kopieren von Schlüsseln entstanden, der aber nicht abgezogen wurde.
Der Schatzmeister, der diese Schlüsselkopie vorgenommen hatte, versprach, diesen
Fehlbetrag auszugleichen. Grundsätzlich sehen die Kassenprüfer aber keine
Bedenken und empfehlen, den Schatzmeister zu entlasten.

Johannes Starosta schlägt an dieser Stelle die Regelung vor, dass nur der
Schatzmeister und bestimmte einkaufsberechtigte Mitglieder Geld im Namen des
Vereins ausgeben dürfen, um die Komplexität zu verringern. In einem Verein, bei
dem er im Vorstand sitzt, sei dies so geregelt.

\emph{[Daniel Sturm erscheint verspätet und wird akkreditiert, insgesamt 25
anwesende stimmberechtigte Mitglieder.]}

%%%%%%%%%%%
\subsection{Entlastung des Vorstands}
\vote{Entlastung des Vorstandes}{19}{0}{5}
Über die Entlastung des Vorstandes wird per Handzeichen abgestimmt. Es stimmen
19 Mitglieder für die Entlastung, 5 Mitglieder enthalten sich, es gibt keine
Gegenstimmen. Die Mitgliederversammlung stellt damit den Vorstand von allen
Ansprüchen frei.

%%%%%%%%%%%
%% TOP 2 %%
%%%%%%%%%%%
\section{Vorstandswahlen}
Als Wahlleiter wird Johannes Starosta einstimmig per Handzeichen gewählt.

Es wird folgendes Wahlverfahren vorgeschlagen: Jedes Mitglied darf für jeden
Kandidaten genau eine Stimme abgeben (Zustimmungswahl), der Kandidat mit den
meisten Stimmen wird gewählt, wenn er mehr als die Hälfte der abgegebenen
Stimmen auf sich vereinen konnte. Bei Stimmgleichheit gibt es eine Stichwahl.

Das Wahlverfahren wird einstimmig angenommen.

%%%%%%%%%%%
\subsection{Wahl des Vorstandsvorsitzenden}
Als Kandidaten für das Amt des Vorstandsvorsitzenden stehen zur Verfügung:
\begin{itemize}
  \item Vincent Breitmoser (Valodim)
  \item Roland Hieber (rohieb)
\end{itemize}

Als Wahlhelfer für diesen Wahlgang melden sich Julien Deseke und Tobias Heine,
es gibt keine Einwände dagegen.

Der Wahlleiter eröffnet den Wahlgang um 16:03. Es wurden 25 Stimmzettel
ausgegeben.

Der Wahlgang wird um 16:12 geschlossen, es wurden 25 Stimmzettel zurückerhalten.
Die Auszählung ergibt folgendes Ergebnis:

\elected{Vorstands\-vorsitzender}{Vincent Breitmoser}{23}{25}
\begin{itemize}
  \item Vincent Breitmoser: 23 Stimmen (92\% der abgegebenen Stimmen)
  \item Roland Hieber: 15 Stimmen (65\% der abgegebenen Stimmen)
\end{itemize}

Vincent Breitmoser nimmt die Wahl an.

\emph{[Juliane Schmidt verlässt die Versammlung und überträgt die Abgabe ihrer
ausgefüllten Stimmzettel auf Jan Lübbe. Es sind noch 24 stimmberechtigte
Mitglieder im Raum.]}

%%%%%%%%%%%
\subsection{Antrag: Übertragung des Stimmrechts regeln}
ktrask stellt spontan mündlich den Antrag, darüber zu entscheiden, ob bei dieser
Versammlung eine Übertragung der Stimmabgabe grundsätzlich möglich ist, sofern
die Stimmzettel von dem Mitglied, das das Stimmrecht überträgt, selber
ausgefüllt worden sind.

Dagegen wird angeführt, dass vorher nicht angekündigt wurde, dass die
Übertragung des Stimmrechts möglich sei. Auch bei anderen Wahl (Bundestagswahl,
Landtagswahl, etc.) sei die Übertragung des Stimmrechts grundsätzlich nicht
zulässig. Allerdings geht es hier nur um die Stimmabgabe, das das Mitglied seine
Stimmzettel vorher selbst ausgefüllt hat.

\vote{Übertragung des Stimmrechts möglich}{11}{7}{5}
Es wird per Handzeichen abgestimmt. 11 Mitglieder sind dafür, eine Übertragung
der Stimmabgabe zu erlauben, 7 Mitglieder stimmen dagegen, 5 Mitglieder
enthalten sich. Der Antrag gilt mit einfacher Mehrheit als angenommen, die
Übertragung des Stimmabgabe an ein anderes Mitglied ist bei dieser Versammlung
möglich, unter der Bedingung, dass das übertragende Mitglied seine Stimmzettel
selbst ausfüllt.

%%%%%%%%%%%
\subsection{Wahl des Stellvertretenden Vorsitzender}
Als Kandidaten für das Amt des stellvertretenden Vorsitzenden melden sich:
\begin{itemize}
  \item Roland Hieber (rohieb)
  \item René Stegmaier (reneger)
  \item Lars Andresen (larsan)
\end{itemize}

Als Wahlhelfer für diesen Wahlgang melden sich Julien Deseke, Julian Kassat und
Tobias Heine, es gibt keine Einwände dagegen.

Der Wahlleiter eröffnet den Wahlgang um 16:20. Es wurden 25 Stimmzettel
ausgegeben.

Der Wahlgang wird um 16:23 geschlossen, es wurden 25 Stimmzettel zurückerhalten,
von denen ein Stimmzettel ungültig war. Die Auszählung ergibt folgendes
Ergebnis:

\elected{Stellvertretender Vorsitzender}{Roland Hieber}{20}{24}
\begin{itemize}
  \item Roland Hieber: 20 Stimmen (83{,}3\% der abgegebenen Stimmen)
  \item René Stegmaier: 5 Stimmen (20{,}8\% der abgegebenen Stimmen)
  \item Lars Andresen: 20 Stimmen (83{,}3\% der abgegebenen Stimmen)
\end{itemize}

Lars Andresen tritt von der Wahl zurück, Roland Hieber nimmt die Wahl an.

%%%%%%%%%%%
\subsection{Wahl des Schatzmeisters}
Als Kandidaten für das Amt des Schatzmeisters melden sich:
\begin{itemize}
  \item Chris Fiege (chrissi\textasciicircum)
  \item Steffen Arntz (DooMMasteR)
\end{itemize}

Als Wahlhelfer für diesen Wahlgang melden sich Julien Deseke, Julian Kassat und
Tobias Heine, es gibt keine Einwände dagegen.

Der Wahlleiter eröffnet den Wahlgang um 16:28. Es wurden 25 Stimmzettel
ausgegeben.

Der Wahlgang wird um 16:31 geschlossen, es wurden 23 Stimmzettel zurückerhalten.
Die Auszählung ergibt folgendes Ergebnis:

\elected{Schatzmeister}{Chris Fiege}{23}{23}
\begin{itemize}
  \item Chris Fiege: 23 Stimmen (100\% der abgegebenen Stimmen)
  \item Steffen Arntz: 5 Stimmen (21{,}7\% der abgegebenen Stimmen)
\end{itemize}

Chris Fiege ist nicht anwesend, aber erklärt fernmündlich gegenüber dem
Wahlleiter, dass er die Wahl zum Schatzmeister annimmt.

%%%%%%%%%%%
\subsection{Wahl der Beisitzer}
Es wird das selbe Wahlverfahren wie bei den vorigen Wahlgängen angewandt, mit
dem Zusatz, dass höchstens die drei Kandidaten mit den meisten Stimmen gewählt
werden, sofern sie mehr als die Hälfte der abgegebenen Stimmen erhalten haben.

Als Kandidaten für den Posten der Beisitzer stellen sich zur Verfügung:
\begin{itemize}
  \item Lena Maria Schimmel
  \item Julian Kassat (omrphuim)
  \item Julien Deseke (Neo Bechstein)
  \item Rebecca Husemann (Pecca)
  \item Matthias Uschok (hellfyre)
  \item Jonas Martin (lichtfeind)
  \item Lars Andresen (larsan)
  \item René Stegmaier (reneger)
\end{itemize}

Als Wahlhelfer für diesen Wahlgang meldet sich Tobias Heine, es gibt keine
Einwände dagegen.

Der Wahlleiter eröffnet den Wahlgang um 16:39. Es wurden 25 Stimmzettel
ausgegeben.

Der Wahlgang wird um 16:41 geschlossen, es wurden 24 Stimmzettel zurückerhalten.
Die Auszählung ergibt folgendes Ergebnis:

\elected{Beisitzer}{Lars Andresen}{21}{25}
\elected{Beisitzer}{Julien Deseke}{20}{25}
\elected{Beisitzerin}{Rebecca Husemann}{18}{25}
\begin{itemize}
  \item Lena Maria Schimmel: 16 Stimmen (66{,}7\% der abgegebenen Stimmen)
  \item Julian Kassat: 8 Stimmen (33{,}3\% der abgegebenen Stimmen)
  \item Julien Deseke: 20 Stimmen (83{,}3\% der abgegebenen Stimmen)
  \item Rebecca Husemann: 18 Stimmen (75\% der abgegebenen Stimmen)
  \item Matthias Uschok: 13 Stimmen (54{,}2\% der abgegebenen Stimmen)
  \item Jonas Martin: 9 Stimmen (37{,}5\% der abgegebenen Stimmen)
  \item Lars Andresen: 21 Stimmen (87{,}5\% der abgegebenen Stimmen)
  \item René Stegmaier: 9 Stimmen (37{,}5\% der abgegebenen Stimmen)
\end{itemize}

Julien Deseke und Lars Andresen nehmen die Wahl an. Rebecca Husemann erklärt die
Annahme der Wahl ferndmündlich gegenüber dem Wahlleiter.

%%%%%%%%%%%
%% TOP 3 %%
%%%%%%%%%%%
\section{Änderungsanträge}

%%%%%%%%%%%
\subsection{Satzungsänderung: Einladung zu Vorstandssitzungen}

rohieb stellt den Antrag, die Satzung wie folgt zu ändern: §8 Abs.~6 Satz~1 mit
der geltenden Formulierung
\begin{quote}
  Die Einladung zu Vorstandssitzungen erfolgt durch den Vorstandsvorsitzenden
  oder den stellvertretenden Vorsitzenden in Textform unter Einhaltung einer
  Einladungsfrist von mindestens 7 Tagen.
\end{quote}
soll ersetzt werden durch die Formulierung
\begin{quote}
  Die Einladung zu Vorstandssitzungen erfolgt durch ein Mitglied des Vorstands
  in Textform unter Einhaltung einer Einladungsfrist von mindestens 7 Tagen.
\end{quote}

Als Begründung wird vom Antragsteller angeführt, dass kein Grund für diese
Einschränkung besteht. Außerdem ist die Wahrscheinlichkeit, dass beide
Vorsitzenden gleichzeitig verhindert sind, größer, als dass alle
Vorstandsmitglieder gleichzeitig verhindert sind.

\vote{Satzungs\-änderung: Einladung zu Vorstandssitzungen durch beliebige
Vorstandsmitglieder erlauben}{22}{0}{1}
Es wird per Handzeichen abgestimmt. Für den Antrag stimmen 22 Mitglieder (88\%
der bei Eröffnung anwesenden Mitglieder), es gibt keine Gegenstimmen und eine
Enthaltung. §8 Abs. 6 Satz 1 der Satzung wird auf die neue Formulierung
geändert.

%%%%%%%%%%%
\subsection{Satzungsänderung: Bestätigung des Vorstands ermöglichen, Amtszeit
bis zur Neuwahl beschränken}

rohieb stellt den Antrag, die Satzung wie folgt zu ändern: §7 Abs.~2, Satz~3 und
4 mit dem geltenden Wortlaut
\begin{quote}
  Die Wiederwahl der Vorstandsmitglieder ist möglich. Die jeweils amtierenden
  Vorstandsmitglieder bleiben nach Ablauf ihrer Amtszeit im Amt, bis Nachfolger
  gewählt sind.
\end{quote}
soll ersetzt werden durch die Formulierung
\begin{quote}
  Die Bestätigung des Vorstandes oder die Wiederwahl der Vorstandsmitglieder ist
  möglich. Die jeweils amtierenden Vorstandsmitglieder bleiben im Amt, bis
  Nachfolger gewählt sind.
\end{quote}

Als Begründung für den Antrag führt er an, dass die Bestätigung des kompletten
Vorstandes als Blockwahl gilt, was nach neuerer Rechtsprechung nicht zulässig
ist, sofern dies nicht in der Satzung verankert ist.\footnote{siehe auch 
\url{https://stratum0.org/vereinsrecht.de-vorstand-blockwahl}}.
Außerdem könnte die Beschränkung der Amtszeit ein Problem sein, falls vorzeitige
Neuwahl eines Vorstandspostens stattfinden sollte.

Es wird die Frage gestellt, ob die neue Formulierung einen Widerspruch zu §8
Abs.~2 Satz~1 der Satzung darstellt. Dies ist nicht der Fall, da es hier nur um
die Übergangszeit zwischen Ämtern geht, in denen die entsprechende Entität nicht
mehr als gewählt gilt.

\emph{[Daniel Sturm verlässt die Versammlung, 22 stimmberechtigte Mitglieder
anwesend.]}

\vote{Satzungs\-änderung: Bestätigung des Vorstands, Amtszeit bis Neuwahl
beschränken}{17}{0}{5}
Die Abstimmung über den Antrag findet per Handzeichen statt. Für den Antrag
stimmen 17 Mitglieder (77{,}2\% der anwesenden Mitglieder) stimmen für den
Antrag, kein Mitglied stimmt dagegen, 5 Mitglieder enthalten sich. Der
Antrag wird angenommen, die Satzung wird auf die neue Formulierung angepasst.

%%%%%%%%%%%
\subsection{Satzungsänderung: Reguläre Amtszeit für Nachfolger von vakant
gewordenen Vorstandsposten}

rohieb stellt den Antrag, §7 Abs.~10, Satz~1 der Satzung mit dem geltenden
Wortlaut
\begin{quote}
  Für vakant gewordene Vorstandsposten wird auf der nächsten
  Mitgliederversammlung jeweils ein Nachfolger bestimmt, der für die restliche
  Dauer der Amtszeit seines Vorgängers im Amt bleibt.
\end{quote}
durch folgenden Wortlaut zu ersetzen:
\begin{quote}
  Für vakant gewordene Vorstandsposten muss auf der nächsten
  Mitgliederversammlung jeweils ein Nachfolger bestimmt werden.
\end{quote}

Dadurch soll verhindert werden, dass folgende Situation eintritt: Ein
Vorstandsmitglied (z.~B. Schatzmeister) tritt vier Wochen vor Ende der Amtszeit
zurück. Nach Einberufung einer Mitgliederversammlung wird zwei Wochen später ein
Nachfolger bestimmt. Dieser Nachfolger hätte dann nur eine Amtszeit von zwei
Wochen, danach müsste wieder neu gewählt werden.

Als Gegenargument wird angeführt, dass durch die neue Regelung mehrere,
gegeneinander verschobene Amtszeiten auftreten können, was bis zu 6
Mitgliederversammlungen im Jahr nach sich ziehen könnte.

\vote{Reguläre Amtszeit für nachgewählte Vorstandsposten}{0}{6}{15}
Die Abstimmung erfolgt durch Handzeichen. Es stimmen 15 Mitglieder gegen den
Antrag, 6 Mitglieder enthalten sich, kein Mitglied stimmt dafür. Der Antrag ist
somit abgelehnt, die Satzung enthält weiterhin die alte Formulierung.

Nach der Abstimmung wird die Frage gestellt, ob die Formulierung des Antrags
kurzfristig geändert werden kann. Die Versammlungsleitung entgegnet, dass dies
nicht der Fall ist, da Satzungsänderungsanträge mit altem und neuen Wortlaut der
Satzung in der Einladung zur Mitgliederversammlung erscheinen müssen.

%%%%%%%%%%%
\subsection{Satzungsänderung: Geschäftsordnung des Vorstands}
rohieb stellt den Antrag, die Satzung wie folgt zu ändern. §7, Abs.~11 mit der
geltenden Formulierung
\begin{quote}
  Der Vorstand gibt sich eine Geschäftsordnung, worin unter anderem die
  Aufgabenteilung des Vorstandes geregelt wird.
\end{quote}
soll ersetzt werden durch die Formulierung
\begin{quote}
  Der Vorstand gibt sich bei Bedarf eine Geschäftsordnung, worin unter anderem
  die Aufgabenteilung des Vorstandes geregelt wird.
\end{quote}

Als Begründung wurd angeführt, dass bisher ist keine Geschäftsordnung nötig war.

Ein Mitglied erwähnt hierzu, dass auch eine formlose, mündliche Absprache als
Geschäftsordnung gelten kann. Des weiteren wird bemängelt, dass der
Schatzmeister als einzige Person Zugriff auf die Mitgliederliste hatte, und
somit einen Überblick über die ausstehenden Zahlungen. Um den Schatzmeister zu
entlasten, sollte dies in einer Geschäftsordnung anders geregelt werden.

Die Abstimmung durch Handzeichen ergibt folgendes Ergebnis: 5 Mitglieder
(22{,}7\% der anwesenden Mitglieder) sind dafür, die Geschäftsordnung des
Vorstands optional zu machen, 11 Mitglieder vertreten die Meinung, dass eine
Geschäftsordnung zwingend nötig ist, 6 Mitglieder enthalten sich. Da das Quorum
von 75\% Pro-Stimmen nicht erreicht wurde, ist der Antrag abgelehnt. Der
Vorstand muss sich weiterhin um eine Geschäftsordnung kümmern.

%%%%%%%%%%%
\subsection{Änderung der Beitragsordnung: ermäßigter Mitgliedsbeitrag für
Auszubildende}

rohieb stellt den Antrag, Auszubildenden auch den ermäßigten Beitrag zugute
kommen zu lassen. Dazu soll §1 Abs.~2 Satz~1 der Beitragsordnung entsprechend
ergänzt werden, sodass er folgende Formulierung enthält:
\begin{quote}
  Schüler, Studenten, Auszubildende, Empfänger von Sozialgeld oder
  Arbeitslosengeld~II einschließlich Leistungen nach §~22 ohne Zuschläge oder
  nach §~24 des Zweiten Buchs des Sozialgesetzbuchs (SGB~II), sowie Empfänger
  von Ausbildungsförderung nach dem Bundesausbildungsförderungsgesetz (BAföG)
  haben die Möglichkeit, einen ermäßigten Beitrag von 12€ pro Monat zu zahlen.
\end{quote}

Es wird die Frage gestellt, ob Auszubildende als Schüler gelten. Jemand
antwortet, dass ab 18 Jahren keine Berufsschulpflicht besteht, und dass die neue
Formulierung auch zur Klarstellung eingefügt werden sollte.

\vote{Ermäßigter Mitgliedsbeitrag für Auszubildende}{22}{0}{0}
Über den Antrag wird per Handzeichen abgestimmt, er wird einstimmig und ohne
Enthaltungen angenommen.

%%%%%%%%%%%
\subsection{Änderung der Beitragsordnung: Alternative Zahlungsweise des
Mitgliedsbeitrags}

rohieb stellt den Antrag, die Beitragsordnung wie folgt zu ändern: §1 soll
durch einen neuen Abs.~5 mit dem folgenden Wortlaut ergänzt werden:
\begin{quote}
  (5) Der Mitgliedsbeitrag kann ersatzweise in Schokoriegeleinheiten gezahlt
  werden. Eine Schokoriegeleinheit ist äquivalent zu 0,60€ anzusehen.
\end{quote}
Weiterhin soll §4~Abs.~1 geändert werden, sodass er der folgenden Formulierung
entspricht:
\begin{quote}
  Die Aufnahmegebühr beträgt zwei Schokoriegeleinheiten, zu zahlen direkt an ein
  präferiertes Vorstandsmitglied. Wahlweise kann die Aufnahmegebühr in der
  Space-Küche für die Allgemeinheit hinterlegt werden.
\end{quote}

Als Begründung wird vom Antragsteller angeführt, dass zur Zeit ein kritischer
Mangel an Schokoriegeln in den Vereinsräumen herrscht. Durch die neue Regelung
könnte diesem Mangel Abhilfe geschaffen werden.

Ein Mitglied merkt an, dass Schokoriegel teilweise weniger als 0,60€ wert sind,
sodass auf Dauer ein starker Wertverfall des Mitgliedsbeitrages festzustellen
wäre. Des weiteren wird angeführt, dass die Miete für die vereinseigenen
Räumlichkeiten nicht in Schokoriegeleinheiten gezahlt werden können, ebenso auch
nicht Strom, Wasser, Heizung, weitere Nebenkosten, Internetzugang oder sonstige
Ausgaben. Als weiterer Punkt gegen die neue Formulierung von §4~Abs.~1 wird
zurecht angemerkt, dass ein neues Mitglied behaupten könnte, dass es die
Aufnahmegebühr in Schokoriegeleinheiten in der Küche hinterlegt hat, diese aber
sofort verzehrt worden wären; die Entrichtung des Aufnahmebeitrags ist somit
nicht nachprüfbar.

\vote{Alternative Zahlungsweise des Mitgliedsbeitrags}{5}{13}{3}
Die Abstimmung per Handzeichen ergibt, dass 5 Mitglieder der neuen Formulierung
zustimmen, während 13 schokoriegelkritische Mitglieder den Antrag ablehnen. 3
Mitglieder enthalten sich der Stimme. Der Antrag ist somit abgelehnt.

%%%%%%%%%%%
\subsection{Beitragsordnung: Lastschrift}
Der Vorstandsvorsitzende erwähnt an dieser Stelle auch nochmal die Möglichkeit,
den Mitgliedsbeitrag per Lastschrift einzuziehen. Bisher waren die Konditionen
bei der Bank eher ungünstig, sodass wahrscheinlich eher Kosten als Nutzen
entstanden wäre. Grundsätzlich hält er das Lastschriftverfahren aber sinnvoll,
möglicherweise verbessern sich die Konditionen auch nach anerkannter
Gemeinnützigkeit.

Der Schatzmeister erwähnt hier, dass ein Lastschriftrücklauf bei den aktuellen
Konditionen 5€-15€ kosten können. Es werden hier von mehreren Mitgliedern
verschiedene Zahlen zu verschiedenen Dingen genannt (0{,}80€ Kosten pro
Bankeinzug, Gebühr auf 3{,}50€ gedeckelt, aber Bearbeitungsgebühr kommt noch
dazu, \ldots). Zudem wird sich voraussichtlich das Lastschriftverfahren
demnächst mit der Einführung des Einheitlichen Euro-Zahlungsverkehrsraumes
(SEPA) ändern, sodass diese Informationen eigentlich nichts nützen.

\consensus{Lastschrift\-verfahren: nicht nötig, kein Interesse der Mitglieder}
Grundsätzlich stellt sich aber die Frage, ob das Lastschriftverfahren überhaupt
von Mitgliedern gewünscht wird. Inzwischen haben allerdings schon viele
Mitglieder Daueraufträge eingerichtet, ein generelles Meinungsbild ergibt auch,
dass viele Mitglieder kein Interesse am Lastschriftverfahren haben.

%%%%%%%%%%%
%% TOP 4 %%
%%%%%%%%%%%
\section{Umgang mit Social Media}\label{top:socialmedia}
Das Thema Social Media wurde schon auf der letzten Mitgliederversammlung am 8.
Januar angerissen, aber dort nicht ausgeführt. Die Meinung über soziale
Netzwerke war damals gespalten. Mittlerweile hat sich aber der Grundsatz "`Wer
macht, hat Recht"' auch in dieser Hinsicht durchgesetzt, sodass inzwischen in
fast allen großen Netzwerken eine Präsenz für den Verein angelegt wurde. Da dies
vorrangig durch Mitglieder in eigenverantwortlichem Handeln geschehen ist,
stellt sich die Frage, inwiefern dies den Verein nach außen repräsentieren
kann.

\consensus{Arbeitsgruppe für Social Media etc. gründen (chrissi\textasciicircum,
reneger, Terminar, weitere)}
Die Diskussion führt schließlich dazu, dass eine Arbeitsgruppe sich dem Thema
annehmen soll, chrissi\textasciicircum, Terminar, reneger und noch ein paar
andere Mitglieder stellen sich dafür zur Verfügung. Dies wird von mehreren
Seiten positiv aufgenommen, es wird der Arbeitsgruppe auch der Zugriff auf
mehrere soziale Netze angeboten. Des weiteren soll es regelmäßige
Zusammenfassungen von aktuell laufenden Projekten über die Mailingliste bzw. den
Newsletter geben.

%%%%%%%%%%%
%% TOP 5 %%
%%%%%%%%%%%
\section{Mitgliederwerbung}\label{top:mitgliederwerbung}
Reneger stellt die Frage, wie es um Nachwuchs bestellt ist, und wie es dem
Verein in den nächsten Generationen geht. Da der aktuelle harte Kern des Vereins
viele Studenten beinhaltet, die vermutlich irgendwann aus Braunschweig wegziehen
werden, ist zu erwarten, dass die Anzahl der Mitglieder dem entsprechend sinkt.
Neo Bechstein schließt sich dieser Meinung an, und fügt hinzu, dass die
ausgetretenen Mitgliedern bisher fast alle Studenten waren, die nicht mehr in
Braunschweig leben.

\consensus{Mitglieder\-werbung in Arbeitsgruppe auslagern, reneger als
Ansprechpartner}
Es wird der Konsens gefasst, dass das Thema in einer Arbeitsgruppe ausgelagert
wird, reneger kümmert sich darum und stellt sich als Ansprechpartner zu
Verfügung.

Ein Mitglied weist auch nochmal darauf hin, dass noch jede Menge Visitenkarten
und Aufkleber existieren, die gerne von jeder Entität verteilt werden dürfen.

%%%%%%%%%%%
%% TOP 6 %%
%%%%%%%%%%%
\section{GPG-Keysigning}
\postponed
Angesichts der fortgeschrittenen Zeit und der fehlenden Vorbereitung wird das
GPG-Keysigning auf einen anderen Zeitpunkt verschoben. Es bietet sich an, dafür
wieder eine Keysigning-Party zu veranstalten.

%%%%%%%%%%%
%% TOP 7 %%
%%%%%%%%%%%
\section{Genehmigung des Protokolls der Mitgliederversammlung 2012-01-08}
\vote{Genehmigung des Protokolls vom 8. Januar 2012}{17}{0}{4}
Das Protokoll der Mitgliederversammlung am 8. Januar 2012 wurde den Mitgliedern
im Vorfeld dieser Sitzung über die Mailingliste und im Wiki zugänglich gemacht.
Es wird über die Genehmigung des Protokolls abgestimmt. 17 Mitglieder stimmen
für die Genehmigung, 4 Mitglieder enthalten sich, es gibt keine Gegenstimmen.
Das Protokoll der Mitgliederversammlung am 8. Januar 2012 wird genehmigt.

\begin{description}
  \item[Veranstaltung geschlossen] um 17:38.
\end{description}

%%%%%%%%%%%%%
%% Anhänge %%
%%%%%%%%%%%%%
\appendix

%%%%%%%%%%%%%%%%%%%%%%%%
%% Kassenbericht 2012 %%
%%%%%%%%%%%%%%%%%%%%%%%%
\section{Kassenbericht 2012}
\includegraphics[width=\textwidth]{images/Kassenbericht_2012_1.pdf}
\newpage
\includegraphics[width=\textwidth]{images/Kassenbericht_2012_2.pdf}

%%%%%%%%%%%%%%%%%%%%
%% Unterschriften %%
%%%%%%%%%%%%%%%%%%%%
\section{Unterschriften}
\vspace{0.7cm}
\noindent Protokollführer: \hrulefill\hfill\phantom{c}\par
\vspace{0.7cm}
\noindent Vorstandsvorsitzender: \hrulefill\hfill\phantom{c}\par
\vspace{0.7cm}
\noindent Stellv. Vorsitzender: \hrulefill\hfill\phantom{c}\par
\vspace{0.7cm}
\noindent Schatzmeister: \hrulefill\hfill\phantom{c}\par
\vspace{0.7cm}
\noindent Beisitzer: \hrulefill\hfill\phantom{c}\par
\vspace{0.7cm}
\noindent Beisitzer: \hrulefill\hfill\phantom{c}\par
\vspace{0.7cm}
\noindent Beisitzer: \hrulefill\hfill\phantom{c}\par

\end{document}
% vim: set tw=80 et sw=2 ts=2:
